\documentclass{article}
\usepackage{amsmath}
\usepackage[T1]{fontenc}
\usepackage{url}
\title{Detecting Erroneous Use of \texttt{memory} Modifier in Solidity}
\author{Kenji Miyamoto}
\date{\today}
\begin{document}
\maketitle
%%%%%\abstract{}
\section{Introduction}
Solidity is a programming language specialized to develop smart
contracts.  The solodity compiler generates a byte code which is executed
within Ethereum Virtual Machine, a cloud computer consisting of
thousands of physical computers\footnote{7450 nodes as of
17.03.2024. (cf. \texttt{https://ethernodes.org/})}.  A byte code is
deployed on an Ethereum blockchain, and is usable by anybody who pays
sufficient cost for execution in the native currency (eg.~Ether) of
the blockchain.  The execution cost of a smart contract depends on how
much and what kind of opcodes (i.e.~low level computational
instructions) it involves, how much persistent data storage it uses,
and so.  Because of this circumstance, it is crucial for smart
contracts to carry out its tasks by means of cheaper opcodes and fewer
amount of persistent data.  Solidity has modifiers such as
\texttt{memory} and \texttt{storage} to specify which data area a
declared variable of reference-type, such as arrays, strings, structs,
and mappings, occupies, and it determines either the data is ephemeral
or persistent.  In order to reduce the execution cost, unnecessary
uses of persistent data should be replaced by ephemeral data, but on
the other hand, an erronous use of ephemeral data causes an unexpected
loss of data modification.  In this paper, we review the Solidity
language focusing on the difference between ephemeral data area and
persistent data area, and then are going to study a static program
analysis technique to detect potential erroneous uses of ephemeral
data areas.
\section{Solidity overview}
In this section, we give an overview on the Solidity language.
Solidity is a statically typed language which uses ECMAScript-like syntax...

Here we give an example of a Solidity code with an erroneous use of data area which we are going to detect through static analysis.
\begin{verbatim}
// SPDX-License-Identifier: GPL-3.0

pragma solidity >=0.8.2 <0.9.0;

contract Example {

    Person[] data;
    uint256 counter;

    struct Person {
        String name;
        uint256 balance;
    }

    constructor() {
        Person memory p1 = Person("John Doe", 10);
        data.push(p1);
        Person memory p2 = Person("Jane Roe", 25);
        data.push(p2);
    }

    function modify(uint256 idx, uint newBalance) public {
        Person memory p = data[idx];
        p.balance=newBalance;
        counter++;
        // p is ephemeral, and the updated balance does not persist.
        // If counter were not modified, this function might be
        // restricted to be view, hence a warning would come up.
    }

    function retrieve(uint256 idx) public view returns (Person) {
        return data[idx];
    }
}
\end{verbatim}
\section{Core language of Solidity}
This section describes a subset of Solidity's syntax\footnote{\url{https://docs.soliditylang.org/en/latest/grammar.html} for the official information on the Solidity grammar.}.
A Solidity program file \textit{source-unit} is of the following structure.
\begin{align*}
  \textit{source-unit} ::= ( & (\texttt{pragma}~\textit{pragma-token}^+ ) ~|~ \textit{contract-definition})^*
\end{align*}
Here we use \({}^+\) for a repetition at least once, \({}^*\) for a repetition at least 0 time, \(|\) for choice, the parentheses for grouping, the italic face for non-terminating symbols, and the typewriter face for terminating symbols.
A \textit{pragma-token} is a token which can contain any kind of symbol except a semicolon.
In this paper, we assume \textit{pragma-token} is a version specification of the Solidity source code.  We omitted the license declaration, which you
often find at the first line starting with \texttt{// SPDX-License-Identifier:} in a Solidity file.
A \textit{contract-definition} defines a smart contract, and its is of the following structure.
\begin{align*}
 \textit{contract-definition} ~::=~ & \texttt{abstract}^{?}~\texttt{contract} ~\textit{identifier} \\
  & \quad (\texttt{is}~\textit{inheritance-specifier}~(\texttt{,}~\textit{inheritance-specifier})^{*})^? \\
  & \quad \mbox{\tt\{} \textit{contract-body-element}^* \mbox{\tt\}}
\end{align*}
Here we use \({}^?\) for an option, namely, a symbol with it either occurs or not at all.  Note that the curly braces in the typewriter face are terminating token and have occurrences in the actual source code.
\textit{identifier} is for the name of the contract, \textit{inheritance-specifier} is for the identifier of a super contract to inherit from.
\textit{contract-body-element} is of the following structure.
\begin{align*}
  & \textit{contract-body-element} ~::=~ \\
  & ~~  \textit{constructor-definition} ~|~ 
\textit{function-definition} ~|~ \textit{modifier-definition} ~|~ \\
& ~~  \textit{fallback-function-definition} ~|~ 
 \textit{receive-function-definition} ~|~
\textit{struct-definition} ~|~ \\
& ~~  \textit{enum-definition} ~|~ 
\textit{user-defined-value-type-definition} ~|~ \\
& ~~ \textit{state-variable-declaration} ~|~
\textit{event-definition} ~|~ 
\textit{error-definition} ~|~
\textit{using-directive}
\end{align*}
\textit{function-definition} is to define a function, and is of the following structure.
\begin{align*}
  & \textit{function-definition} ~::=~ \\
  & \quad \texttt{function}~(\textit{identifier} ~|~
  \texttt{fallback} ~|~ \texttt{receive} ) ~\texttt{(} \textit{parameter-list}\texttt{)}\\
  & \qquad (\textit{visibility} ~|~ \textit{state-mutability} ~|~ \textit{modifier-invocartion} ~|~ \texttt{virtual} ~|~ \\
  & \qquad \qquad \qquad \textit{override-specifier})^* \\
  & \qquad (\texttt{returns}~\texttt{(} \textit{parameter-list} \texttt{)})^? \\
  & \qquad (\texttt{;} ~|~ \textit{block})
  %%\mbox{\tt\{} \textit{contract-body-element}^* \mbox{\tt\}}
\end{align*}
%% which is a line starting with the text  followed by an identifier
%% such as \texttt{GPL-3.0}, \texttt{MIT}, etc.
%% This line is a comment, as you find \texttt{//}, but the Solidity compiler issues a warning message if this license declaration is not properly given.
%% \texttt{Pragma} is compiler directives.
{\it To do: further description on the syntax.}

\textit{block} and \textit{statement} are given as follows.
\begin{align*}
\textit{block} ~::=~ & \mbox{\tt\{} \textit{statement}^* \mbox{\tt\}} \\
\textit{statement}  ~::=~ & \textit{block} ~|~
\textit{variable-declaration-statement} ~|~
\textit{expression-statement} ~|~ \\
& \textit{if-statement} ~|~
\textit{for-statement} ~|~
\textit{while-statement} ~|~ \\
& \textit{do-while-statement} ~|~
\textit{continue-statement} ~|~
\textit{break-statement} ~|~ \\
& \textit{try-statement} ~|~
\textit{return-statement} ~|~
\textit{emit-statement} ~|~\\
& \textit{revert-statement} ~|~
\textit{assembly-statement}
\end{align*}
As common in modern languages, a block is not just a container of statements but it also creates a variable scope.
\section{Implementation}

\end{document}
